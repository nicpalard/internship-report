\chapter{Conclusion}

Au cours de ce stage, j'ai pu mettre en pratique les nombreux enseignements qui m'ont été dispensés tout au long de mon cursus. En passant de l'architecture logicielle au traitement d'image sans oublier le rendu 3D, j'ai eu l'occasion de tester, d'approfondir et même d'acquérir diverses connaissances comme par exemple la programmation sur carte graphique, l'utilisation d'\texttt{Unity} ou encore le développement d'une architecture en microservices. Immergé dans le domaine de la réalité augmentée spatiale, j'ai pu cerner et comprendre les enjeux, les apports mais aussi les difficultés et les contraintes que représente une telle technologie. En effet, j'ai pu, d'une part, saisir l'importance de la communion entre contenu et interface ainsi que l'importance du contexte lors des développements d'application qui m'ont été confiés (comme la \texttt{ReARTable}, l'extraction de documents ou encore l'illusion de projection). D'autre part, la création du prototype haute performance a été l'occasion d'aborder le côté beaucoup plus technique de cette technologie, comme les différents besoins en puissance de calcul, gestion des ressources et gestion du système.\\

Le stage n'étant pas fini, le dernier mois de celui-ci sera consacré à poursuivre le développement et l'amélioration du module \texttt{Unity} ainsi que du prototype haute performance réalisés, afin de pallier aux divers problèmes encore présents et évoqués tout au long de ce mémoire (latence caméra, autres optimisations, nouveau pipeline événementiel, gestion des clés dans \texttt{Unity}). En plus de ces développements, il sera aussi question de rédiger davantage de tests afin de permettre à la société de déterminer, en se basant non pas uniquement sur des tests d'utilisation mais également sur des tests de performance générale du système, si elle souhaite continuer à développer cette technologie ou si les contraintes techniques entourant ce domaine sont encore trop fortes.\\

Outre les enrichissements en termes de connaissances et de compétences j'ai aussi eu l'occasion de bénéficier d'une grande ouverture professionnelle en travaillant dans un environnement totalement différent de tout ce que j'avais connu auparavant où la liberté, l'autonomie et la volonté de créer sont des valeurs fortement mises en avant.
J'ai d'ailleurs, grâce à ce stage, pu décrocher un contrat de travail dans le domaine de la réalité augmentée spatiale dont la mission principale sera le développement d'outil de création de contenu pour les déficients visuel à l'\texttt{Inria}.