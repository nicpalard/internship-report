\chapter{Conclusion}

Tout au long de ce stage, j'ai pu mettre en pratique les nombreux enseignements qui m'ont été dispensé tout au long de mon cursus. En passant par de l'architecture logiciel au traitement d'image sans oublier le rendu 3D, j'ai réellement eu l'occasion de tester et d'approfondir et même d'acquérir des connaissances. Immergé dans le domaine de la réalité augmentée, j'ai pu cerner et comprendre les enjeux, les apports mais aussi les difficultés et les contraintes que représente une telle technologie. Aussi bien lors des divers développement d'applications qui m'ont été confié où j'ai pu comprendre l'importance de la communion entre contenu et interface que lors du développement de la technologie elle même comme lors de la création du prototype autre performance où j'ai eu l'occasion d'aborder le côté beaucoup plus technique de celle ci. \\

Le stage n'étant pas fini le dernier mois de celui ci sera consacré à continuer et à améliorer le module Unity ainsi que le prototype haute performance réalisés afin de pallier aux divers problèmes encore présents qui ont été évoqués. En plus de ces développements, il sera aussi question de rédiger plus de tests afin que la société puisse déterminer, en se basant non pas juste sur des résultats d'utilisation, si elle souhaite continuer à développer cette technologie ou si les contraintes techniques sont encore trop forte.\\

Outre les enrichissements en terme de connaissances et de compétence j'ai aussi eu l'occasion d'avoir un très grand enrichissement professionnel en travaillant dans un environnement totalement différent de tout ce que j'avais connu auparavant ou la liberté, l'autonomie et la volonté de créer sont des valeurs fortement mises en avant.
J'ai d'ailleurs, grâce à ce stage, pu décrocher un contrat de travail dans le domaine de la réalité augmentée dont la mission principale sera le développement d'outil de création de contenu en réalité augmentée pour les déficients visuel.