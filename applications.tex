\chapter{Développement d'application}

Comme expliqué dans l'introduction chapitre ~\ref{chap:intro} le premier objective du stage était le développement d'applications de réalité augmentée spatiale.Il était important pour commencer le stage d'évaluer les possibilités offertes mais aussi les contraintes posés par le kit de développement. Le travail demandé ne se cantonnais donc pas uniquement au développement d'application mais il était important d'effectuer un travail d'analyse et de critique du kit.

\section{ReARTable}
D'après le contexte et le public ciblé par l'entreprise, il m'a paru intéressant de développer une démonstration à but à la fois ludique et éducatif. J'ai donc choisit de recréer la technologie Reactable\cite{reactable} proposé par la société du même nom.

La Reactable est un instrument de musique electronnique permettant la génération de son en direct a l'aide d'éléments

% Analyse de technologie de la reactable
	% Retro eclairage -> marqueur fiduciaire -> interface tangible -> visualisation très poussés -> très cher
% Analyse rapide des besoins de l'application
	% Pouvoir créer différents effets sonores
	% Pouvoir jouer plusieurs effet sonores en même temps (cohérence des loops)
	% Apporter du controle aux effets sonores
	% Visualiser les effets sonores
	% Visualiser le son
% Proposition schématique de comment adapter la reactable au système actuel
% Proposition des modes d'interactions en adéquation avec le concept d'interface tangibles
% Parler de la visualisation (schéma expliquatifs)
% Parler du résultat final et des soucis d'implémentation
% Parler des cas d'utilisations

\section{Extraction de document}

% Analyse rapide des besoins de l'application
% Proposition schématique de toutes les techniques misent en oeuvre (expliquer la notion d'echelle et de connaissance a priori)
% Parler du résultat final et des soucis d'implémentation
% Parler des cas d'utilisation