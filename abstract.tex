\renewcommand{\abstractnamefont}{\normalfont\Large\bfseries}
%\renewcommand{\abstracttextfont}{\normalfont\Huge}

\begin{abstract}
\hskip7mm

\begin{spacing}{1.3}

Dans le cadre de mon stage de fin d'étude, j'ai eu l'occasion de travailler chez \texttt{RealityTech}, une jeune startup dans le domaine de la réalité augmentée spatiale. Durant le temps que j'y ai passé, il m'a été demandé d'étudier les objectifs, les intérêts, et les apports de cette technologie. J'ai tout d'abord pu l'apprivoiser par le biais de \text{PapARt}, un système de projection interactif, avec lequel j'ai développer mes premières applications. Tout au long de ces développements j'ai découvert les apports, les enjeux mais aussi les difficultés inhérentes à de tels systèmes. Ces diverses difficultés ainsi que la rigidité de \texttt{PapARt} m'ont amené à développer un nouveau prototype de plateforme pour la société dont le but était d'offrir de meilleures performances que les systèmes actuels. De l'optimisation matérielle à l'optimisation de l'architecture en passant par l'optimisation algorithmique, il m'a été demandé d'opérer sur tous les fronts afin de créer un prototype viable. Un prototype n'allant pas sans tests, j'ai du réaliser différents tests de performance qui ont permis de mettre en lumière les avancées et les défauts de ce dernier. Les performances n'étant pas le seul critère de validation du prototype, j'ai poursuivi mon stage en développant un module Unity permettant de le mettre a profit. En plus de permettre l'évaluation de l'utilisabilité du prototype, le module devait aussi fournir aux utilisateurs développeurs de \texttt{RealityTech} la possibilité de créer des applications de réalité augmentée spatiale en résolvant pour ces derniers les nombreuses problématiques liés au domaine. Ainsi, pour terminer mon stage, j'ai endossé le costume de l'utilisateur développeur et ai réaliser une application de démonstration en utilisant le module nouvellement crée afin d'en faire l'évaluation.
\end{spacing}
\end{abstract}
