\renewcommand{\abstractnamefont}{\normalfont\Large\bfseries}
%\renewcommand{\abstracttextfont}{\normalfont\Huge}

\begin{abstract}
\hskip7mm

\begin{spacing}{1.3}

Dans le cadre de mon stage de fin d'étude, j'ai eu l'occasion de travailler chez \texttt{RealityTech}, une jeune startup dans le domaine de la réalité augmentée spatiale. Durant le temps que j'y ai passé, il m'a été demandé d'étudier les objectifs, les intérêts et les apports de cette technologie. J'ai tout d'abord pu l'apprivoiser par le biais de \texttt{PapARt}, le système de projection interactif que développe la société, avec lequel j'ai développé mes premières applications. Tout au long de ces développements, j'ai découvert les enjeux mais aussi les difficultés inhérentes à de tels systèmes. Ces diverses difficultés ainsi que la rigidité de \texttt{PapARt} m'ont amené à développer un prototype d'une nouvelle plateforme pour la société, dont le but était d'offrir de meilleures performances et d'être plus robuste que le système actuel, tout en en conservant les fonctionnalités. De l'optimisation matérielle à la refonte complète de l'architecture, en passant par l'optimisation algorithmique, il m'a été demandé d'opérer sur tous les fronts afin de créer un prototype viable. Un prototype n'allant pas sans tests, j'ai du m'atteler à réaliser une batterie de tests de performance qui ont permis de mettre en lumière les avancées mais aussi les défauts de ce dernier. Les performances n'étant pas le seul critère de validation du prototype, j'ai poursuivi mon stage en développant un module Unity permettant de le mettre à profit. En plus de permettre l'évaluation de l'utilisabilité du prototype, le module devait aussi fournir aux utilisateurs développeurs de \texttt{RealityTech} la possibilité de créer des applications de réalité augmentée spatiale en résolvant, pour ces derniers, les nombreuses problématiques liées au domaine. Ainsi, pour terminer mon stage, j'ai endossé le costume de l'utilisateur développeur et ai réalisé une application de démonstration en utilisant le module nouvellement créé afin d'en réaliser l'évaluation.\\

\textit{Mot-clés:} Réalité augmentée, réalité augmentée spatiale, interface tangible, programmation haute performance, microservices, unity, processing, vision par ordinateur.

\end{spacing}
\end{abstract}
