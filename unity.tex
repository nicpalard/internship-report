\chapter{Unity}

Dans cette partie on parlera finalement du developpement du module Unity utilisant la nouvelle plate-forme mise en place au cours du stage dont il est question Chapitre~\ref{chap:protoHP}. Ce développement a été la parfaite occasion de tester et mettre a l'épreuve cette dernière et avoir un retour réel sur son utilisabilité.
Il sera discuté, dans un premier temps, des enjeux, des apports, de la cible et de la conception du module en abordant rapidement les difficultés rencontrés lié a Unity ou a la nouvelle plate-forme puis dans un second temps du développement d'application avec ce module.

Pour finir, nous effectuerons une rapide comparaison entre la version actuelle et la version en développement des kit de développement (Unity et Processing) afin d'avoir une évaluation dans des conditions réelles d'utilisation.

\section{Module Unity}

% Connection a redis, reconstruction des images, protocol de transfert, affichage dans l'éditeur, execution en mode édition, intégration du control de nectar

\section{Application Unity}