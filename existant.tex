\chapter{État de l'art}

Intro

\section{PapARt}
\label{sec:papart}
Comme décrit dans l'introduction chapitre~\ref{chap:intro}, PapARt ou Paper Augmented Reality Toolkit se présente sous la forme d'un kit de développement permettant de créer des applications interactives en réalité augmentée de création de dessins ou de peinture (fig~\ref{fig:papartdemo}). L'idée est de proposer une technique numérique non intrusive pour faciliter une tâche complexe, tel que le dessin tout en permettant à l'utilisateur de s'exprimer. % Parler de la conservation des proportions

\begin{figure}[H]
\centering
\includegraphics[width=0.65\textwidth]{images/papart-demo}
\caption{Jeremy Laviole utilisant PapARt pour dessiner en réalité augmentée\protect\footnotemark}
\label{fig:papartdemo}
\end{figure}
\footnotetext{Source: \href{https://team.inria.fr/potioc/fr/scientific-subjects_old/papart/}{Inria - PapARt}}

Le système interactif (fig~\ref{fig:papartsystem}) permettant d'utiliser tout le potentiel de PapARt est très spécifique et se compose de, 2 dispositifs d'acquisitions.

Le premier est une caméra couleur observant la zone de travail dont le but est de détecter les feuilles de papier servant de base à la projection. Ces feuilles sont ornés de marqueurs ARToolKitPlus\footnote{\href{https://github.com/paroj/artoolkitplus}{https://github.com/paroj/artoolkitplus}}, qui est une bibliothèque de détection de marqueurs fiduciaires \footnote{Un marqueur fiduciaire est un objet placé dans le champ de vision le plus souvent de système d'imagerie qui apparait sur l'image produite et qui va servir de point de repère ou de référence.}, qui ont deux utilisations. Ils permettent a la fois la détection de la feuille de papier, en effet, avec un nombre suffisants de marqueurs détectés et une connaissance a priori du modèle de la feuille il est possible d'en estimer la pose \footnote{\href{https://en.wikipedia.org/wiki/Pose_(computer_vision)}{Pose : Wikipédia}} (position et rotation dans l'espace), et à la fois la détermination du contenu de la feuille. Chaque marqueurs étant unique, lorsque PapARt détecte une feuille de papier, il récupère également les informations des marqueurs associés permettant de projeter du contenu différents en fonction des marqueurs présents.

Le deuxième dispositif d'acquisition est une caméra de profondeur dont rôle est de détecter les différents utilisateurs, les différents objets et les potentielles interactions. Grâce aux informations de profondeur, les interactions peuvent être détecté soit sur le plan de la zone de travail, ce sont des interactions qualifiées de "touch", soit dans l'espace au dessus de la zone de travail, qu'on qualifiera de "pointage 3D".

Pour finir, en plus des deux dispositifs d'acquisition, un dispositif de projection est présent pour permettre la visualisation du contenu. Son rôle est de projeter dans les zones adéquates (i.e détectées via des feuilles de marqueur) le contenu numérique désiré.

Pour que le projecteur soit capable de projeter précisément des informations dans les feuilles détectées par la caméra, une calibration très précise ainsi qu'une représentation a l'échelle est nécessaire. C'est aussi cette calibration qui permet au système d'avoir des capacités d'interaction (toucher simple, toucher multiple, balayage et autre) sensiblement similaires à celle d'une tablette tactile. 
%TODO figure

\begin{figure}[H]
\centering
\includegraphics[width=0.4\textwidth]{images/papart-system}
\caption{Système interactif utilisant PapARt\protect\footnotemark}
\label{fig:papartsystem}
\end{figure}

\footnotetext{Source: \href{https://team.inria.fr/potioc/fr/scientific-subjects_old/papart/}{Inria - PapARt}}

Au delà des applications d'aide au dessins qui sont extrêmement nombreuses, PapARt ouvre un champ des possible assez large. Le rôle de RealityTech est d'explorer ce champ des possibles en améliorant PapARt et en fournissant de nouveau cas d'utilisation toujours plus innovants.

% Mettre des exemples


\section{Systèmes de réalité augmentée spatiale}
\label{sec:SARother}
\paragraph{RoomAliveToolKit} RoomAliveToolKit\cite{Jones:2014:RME:2642918.2647383} est un projet tout droit sorti des laboratoires de recherche de Microsoft. RoomAliveToolKit est un kit de développement créé en 2013 par Nikunj Raghuvanshi, Eyal Ofek et Andy Wilson qui permet, a l'instar de PapARt, de créer des expériences de projection interactive. La principale différence réside dans le fait que RoomAliveToolKit a pour but de donner vie à des pièces entière en utilisant plusieurs projecteurs et plus caméra qui fonctionne a l'unisson.

RoomAliveToolKit a permis entre autre de développer de nombreux projets basés sur de la projection interactive tel que RoomAlive, Room2Room, IllumiRoom et bien d'autre.

\begin{figure}[H]
    \centering
	\subfloat[Système de projection interactive nécessaire a l'utilisation de RoomAliveToolkit]{
      \includegraphics[width=0.45\textwidth]{images/roomalivesystem}
      \label{sub:roomalivesystem}
      }
    \subfloat[RoomAlive - Démonstration]{
      \includegraphics[width=0.45\textwidth]{images/roomalivedemo}
      \label{sub:roomalivedemo}
      }
\caption{Microsoft Research: RoomAliveToolkit et RoomAlive\protect\footnotemark}
\label{fig:roomalive}
\end{figure}
\footnotetext{Source: \href{https://github.com/Microsoft/RoomAliveToolkit}{RoomAliveToolkit}}


% Parler de la mise en place de la coordination de plusieurs projecteurs, des illusions projectives, 

\section{Bilan}