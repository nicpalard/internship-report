\chapter{Introduction}
\label{chap:intro}

Ce mémoire retracera les missions réalisées durant mon stage de Master 2 Informatique pour l'Image et le Son à l'Université de Bordeaux 1 effectué entre Avril et Septembre 2018 (6 mois) dans la société RealityTech. Ce rapport ne couvrira cependant que les 5 premiers mois du stage car la date de rendu de ce dernier précède d'un moi la date de fin du stage.\\

Le stage a donc été effectué chez RealityTech une jeune start-up de réalité augmentée spatiale. Issue de l'Inria de Bordeaux, l'institut national de la recherche en information et en automatique, cette dernière est la continuité d'un projet de recherche mené par Jérémy Laviole, ex ingénieur de recherche à l'Inria. Ce projet, PapARt\footnote{https://project.inria.fr/papart/fr/} Paper Augmented Reality ToolKit, est un kit de développement (SDK) permettant de créer des expériences de réalité augmentée sous forme d'applications de projection interactive dans des feuilles de papier. Pour créer ces expériences, du matériel spécifique est nécessaire, des caméra couleurs et des caméra de profondeur utilisées pour capter le monde réel, un projecteur pour pouvoir projeter le contenu numérique et un ordinateur. Ce matériel est fournis par RealityTech qui réalise tout le travail de calibration et d'installation de l'environnement nécessaire. En parallèle de la partie matériel, la société continue de développer activement le SDK et de créer des applications, afin d'étendre les possibilités de la technologie. Le but est de pouvoir ouvrir la technologie a d'autre domaines que celui de la recherche pour mettre les outils collaboratifs, les systèmes interactifs aux services de l'éducation, du vidéo ludique, de la vulgarisation etc...
%Les travaux actuellement effectués a RealityTech visent à améliorer et étendre ce système de projection. Le but est de pouvoir créer, via ce que propose la société, des expériences collaboratives où les objets physiques se mêlent parfaitement au monde numérique que ce soit en créant des interactions avec ceux ci, ou en leur rajoutant du contexte.\\

\section{Contexte et cadre}
\label{sec:contexte}
Actuellement, RealityTech se développe dans un incubateur de start-up appelé \texttt{La Banquiz}\footnote{http://labanquiz.com} situé 4 rue Eugène et Marc Dulout, à Pessac Centre. L'objectif de La Banquiz est de promouvoir des start-up Open Source\footnote{https://fr.wikipedia.org/wiki/Open\_source} et innovantes en leur apportant a leur dirigeant des formations, du coaching individuel et collectif, de l'aide pour la recherche de financement et tout ce qui gravite autour de l'accompagnement de jeunes entreprises.\\

A ce jour RealityTech ne travail qu'avec des laboratoire de recherche (comme ...) et cherche a étendre son secteur d'activité. Les systèmes proposés par la société fournissent les résultats espérés et la dynamique de celle ci s'oriente donc vers une commercialisation du produit. Dans cette optique, l'entreprise souhaite innover du point vue matériel et logiciel en développant une plateforme haute performance mais aussi du point de vue utilisabilité. Actuellement le développement d'application clients se fait via un SDK Processing\footnote{} cependant ce dernier ne réponds pas totalement aux besoins lorsqu'il s'agit de développer des applications utilisant par exemple du la 3D.\\

Pour débuter ce stage, j'ai eu l'occasion de partir au Laval Virtual, le plus grand salon international sur la réalité augmentée et virtuelle, pendant une semaine en tant qu'exposant. J'ai pu développer une bonne connaissance du produit et ai pu observer de nombreuses technologies à l'état de l'art dans ces domaines. C'est aussi ce salon qui m'a permis de comprendre les besoins auxquels pouvait répondre une technologie comme celle de RealityTech et par la même occasion les enjeux et les apports de celle ci. Ça a aussi été une très bonne expérience au niveau relationnel car elle à permis de créer rapidement une relation avec Jeremy Laviole, mon tuteur de stage.

Le reste de mon stage a été réalisé à la Banquiz ou j'ai été au contact de toutes les entreprises officiant en son sein. J'ai notamment pu rencontrer d'autres stagiaires tel que Rémi Kressmann, développeur, et Gabin Andrieux, commercial chez LockEmail, une entreprise de cyber-sécurité, mais aussi des dirigeants comme Jean François Schaff, un brilliant physicien ayant créer une plateforme de télé expertise pour les professionnels de santé, avec qui j'ai énormément échangé aussi bien sur des concepts de programmation, que sur la culture de l'informatique en générale.

% Expliquer le cadre de travail, incubateur, beaucoup de réunion, de démarchage, besoin d'applications de démonstration, besoin de prototype pour avoir des fonds etc autonomie très importante ..
% Parler du contexte  en terme de logiciel : volonté d'évolué et de partir vers une nouvelle base moins contraignante que celle présente actuellement + volonté d'ouverture au grand public (ce qui a fait que j'ai du dev unity) + modularité
% Peut être parler plus en détail de PapARt (calibration camera projecteur, expliqué qu'on travail avec des caméra, a l'échelle dans le monde réel (beaucoup de problèmes de calibration etc ...)

\begin{center}
Problématique du sujet
\end{center}


\section{Objectifs}
Le déroulement du stage a été fortement guidé par les besoins de la société.

\paragraph{Applications de démonstration} Le premier gros objectif du stage était le développement d'applications de démonstration en utilisant le produit de l'entreprise. Le but était de comprendre l'essence, le fonctionnement global du produit et ce qu'il était possible/impossible de réaliser avec celui ci. Cet objectif m'a permis d'acquérir à la fois une vision globale de l'architecture logiciel et du fonctionnement interne du kit de développement, et de l'architecture matérielle nécessaire a l'utilisation du kit. En développant ces applications de démonstration, j'ai acquis une vision globale du projet qui m'a permis d'avoir une certaine autonomie assez rapidement
%TODO Refaire

\paragraph{Plateforme haute performance} Le deuxième objectif était de réaliser une preuve de concept haute performance du produit. En effet, comme expliqué dans la partie sur le contexte (sec. \ref{sec:contexte}), l'entreprise se lançait dans le développement d'une nouvelle plateforme haute performance. Aussi bien au niveau matériel, ordinateur, caméra, projecteur, que logiciel, algorithmes, communication inter processus, accès au matériel, il a fallut réaliser des tests complets.
%TODO Pas fini

\paragraph{Kit de développement} Le dernier objectif était le développement d'un kit de développement (SDK) permettant de créer des applications de réalité augmentée spatiale ou en vue au travers utilisant la technologie de RealityTech sous Unity. Le but était de pouvoir profiter de la puissance d'Unity en tant que moteur de jeu / moteur 3D. % et aux utilisateurs avancés de pouvoir developper des applications dans un environnement qu'ils connaissent et qui est fait pour ça. Point cruciale car très utilisé aussi bien indé que industrie, c'est un standard qu'il fallait integrer pour se developper.
%TODO pas fini