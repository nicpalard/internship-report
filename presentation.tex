\chapter{Présentation du projet}

Intro\footnotemark\\
%note en bas de page

\section{RealityTech}
RealityTech est une jeune startup de réalité augmentée spatiale. Issue de l'Inria de Bordeaux, l'institut national de la recherche en information et en automatique, cette dernière est la continuité d'un projet de recherche mené par Jérémy Laviole, ex ingénieur de recherche à l'Inria. Ce projet de recherche s'appelait PapARt, Paper Augmented Reality ToolKit et proposait un ensemble de module permettant de créer des expériences de réalité augmentée dans une feuille de papier. Aujourd'hui, le projet a évolué et propose maintenant des expériences de projections interactives.
% Biblio : PapARt, Inria, Reality Tech, Realité Augmentée Spatiale

\section{Contexte}
% Expliquer le cadre de travail, incubateur, beaucoup de réunion, de démarchage, besoin d'applications de démonstration, besoin de prototype pour avoir des fonds etc ..

\begin{center}
Problématique du sujet
\end{center}

\section{Hypothèse de solution}

%Quoi :
Bla\\

Voici une liste :
\begin{itemize}
\item item 1;
\item item 2;
\item item 3;
\item item 4.
\end{itemize}

Bla\\

%Comment :
Bla

Bla\footnotemark\\

%Detail :
Bla(cf. ref. \cite{cite6}).
%citation référencé dans le document "bibliographie.bib" inclus à la fin du document

\footnotetext{Note bas de page "bla"}