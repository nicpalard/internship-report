\chapter{Introduction}
\label{chap:intro}

Ce mémoire retracera les missions réalisées durant mon stage de Master 2 Informatique pour l'Image et le Son à l'Université de Bordeaux 1 effectué entre Avril et Septembre 2018 (6 mois) dans la société \texttt{RealityTech}\footnote{\href{http://rea.lity.tech/}{http://rea.lity.tech/}}. Ce rapport ne couvrira cependant que les 5 premiers mois du stage car la date de rendu de ce dernier précède d'un moi la date de fin du stage.\\

Le stage a donc été effectué chez \texttt{RealityTech} une jeune start-up de réalité augmentée spatiale. Issue de l'Inria de Bordeaux, l'institut national de la recherche en information et en automatique, cette dernière est la continuité d'un projet de recherche mené par M. Laviole, ex ingénieur de recherche à l'Inria. Ce projet, PapARt\footnote{\href{https://project.inria.fr/papart/fr/}{https://project.inria.fr/papart/fr/}} Paper Augmented Reality ToolKit, est un kit de développement (SDK) Processing permettant de créer des expériences de réalité augmentée sous forme d'applications de projection interactive dans des feuilles de papier. Pour réaliser ce type d'application du matériel spécifique est nécessaire tel que des caméra couleurs, des caméra de profondeur qui sont utilisées pour capter le monde réel, un projecteur utiliser pour pouvoir visualiser le contenu numérique dans l'espace et un ordinateur pour effectuer tous les calculs nécessaire au bon fonctionnement des applications. C'est dans un premier temps du côté matériel que \texttt{RealityTech} intervient. En effet, celle ci propose des systèmes pré calibré et prêt à l'emploi résolvant donc tous les problèmes de calibration et d'installation de l'environnement nécessaire à l'utilisation de PapARt. En parallèle de la partie matériel, la société développe et améliore activement le kit et créer aussi diverses applications de démonstration, afin de montrer et d'étendre les possibilités de la technologie. Le but est de pouvoir ouvrir la technologie à d'autres domaines d'activités que celui de la recherche pour mettre les outils collaboratifs et systèmes interactifs aux services de l'éducation, du vidéo ludique, de la vulgarisation etc....

\section{Contexte et cadre}
\label{sec:contexte}
Actuellement, \texttt{RealityTech} se développe dans un incubateur de start-up appelé \texttt{La Banquiz}\footnote{\href{http://labanquiz.com}{http://labanquiz.com}} situé 4 rue Eugène et Marc Dulout, à Pessac Centre. L'objectif de \texttt{La Banquiz} est de promouvoir des start-ups Open Source\footnote{\href{https://fr.wikipedia.org/wiki/Open\_source}{Open Source - Wikipédia}} et innovantes en apportant à leurs dirigeants des formations, du coaching individuel et collectif, de l'aide pour la recherche de financement et tout ce qui gravite autour de l'accompagnement de jeunes entreprises.\\

À ce jour \texttt{RealityTech} ne travail qu'avec des laboratoire de recherche tel que l'Inria et cherche à étendre son secteur d'activité. Les systèmes et services proposés par la société fournissent les résultats espérés et la dynamique de celle ci s'oriente donc vers une industrialisation du produit. Ainsi l'entreprise souhaite passer de prototype FabLab créé à l'unité basé sur des technologies de recherche, à prototype industriel basé sur des technologies populaires dans le monde de l'industrie. \texttt{RealityTech} se lance donc dans la création d'une plateforme haute performance appelé Nectar et dans le développement d'un nouvel SDK Unity\footnote{\href{https://unity3d.com/fr}{https://unity3d.com/fr}} utilisant cette plateforme. Le SDK Processing actuel est très accessible et permet de prototyper bon nombre d'applications rapidement mais il ne permet cependant pas de répondre aux besoins de l'industrie plus spécialement quand il s'agit de développer des applications client ayant besoin d'un moteur 3D, ou d'un moteur physique performant.

Pour débuter ce stage, j'ai eu l'occasion de partir à Laval pendant une semaine pour assister au Laval Virtual, le plus grand salon international sur la réalité augmentée et virtuelle, en tant qu'exposant. Grâce a ce salon, j'ai pu développer une bonne connaissance du produit et ai pu observer de nombreuses technologies à l'état de l'art dans ces domaines. C'est aussi ce salon qui m'a permis de bien comprendre les besoins auxquels pouvait répondre une technologie comme celle de \texttt{RealityTech} et par la même occasion les enjeux et les apports de celle ci. Cela a aussi été une très bonne expérience au niveau relationnel car elle m'a permis de créer rapidement une relation avec M. Laviole, mon tuteur de stage.

Le reste de mon stage a été réalisé à \texttt{La Banquiz} où j'ai été au contact de toutes les entreprises officiant en son sein. J'ai notamment pu rencontrer d'autres stagiaires tel que Rémi Kressmann, développeur, et Gabin Andrieux, commercial chez \texttt{LockEmail}\footnote{\href{http://www.lockemail.fr/}{http://www.lockemail.fr/}}, une entreprise de cyber-sécurité, mais aussi des dirigeants comme Jean François Schaff, docteur en physique quantique ayant créer \texttt{Postelo}\footnote{\href{https://www.postelo.fr/}{https://www.postelo.fr/}} une plateforme de télé expertise pour les professionnels de santé, avec qui j'ai énormément échangé aussi bien sur des concepts de programmation, que sur la culture de l'informatique en générale.

\begin{center}
Problématique du sujet
\end{center}

\section{Objectifs}
Le déroulement du stage a été fortement guidé par les besoins de la société.

\paragraph{Applications de démonstration} Un des objectifs du stage était le développement d'applications de démonstration en utilisant le système de l'entreprise. Le but était de comprendre l'essence, le fonctionnement global du système et ce qu'il était possible/impossible de réaliser avec celui ci. Cet objectif avait pour but d'acquérir à la fois une vision globale de l'architecture logiciel, du fonctionnement interne du kit de développement, et de l'architecture matérielle nécessaire à son utilisation mais aussi de développer une certaine autonomie pour la suite du stage.
%TODO Refaire

\paragraph{Plateforme haute performance} Le deuxième objectif était de réaliser une preuve de concept haute performance du produit. En effet, comme expliqué dans la partie sur le contexte (sec. \ref{sec:contexte}), l'entreprise se lançait dans le développement d'une nouvelle plateforme haute performance. Aussi bien au niveau matériel, ordinateur, caméra, projecteur, que logiciel, algorithmes, communication inter processus, accès au matériel, il a fallut réaliser des tests complets.
%TODO Pas fini

\paragraph{Kit de développement} Comme expliqué dans le cadre, les besoins auxquels répond le kit de développement Processing ne sont plus suffisants lorsqu'il est question d'applications devant faire le rendu de grosse scène 3D, ou des simulation physique,. Le développement d'une version \texttt{Unity} du kit permettant de créer des applications de projection interactive était donc nécessaire pour la suite du développement de \texttt{RealityTech}. Pour que le développement de ce module soit efficace, l'objectif était d'intégrer et d'utiliser les micro services du prototype haute performance Nectar pour gérer tout ce qui concerne le monde physique (caméras, projecteurs, suivi d'objet, événements de toucher, etc) et d'utiliser Unity dans son rôle de moteur 3D, moteur physique pour gérer le rendu de la scène, la physique des objets dans la scène, la lumière et créer des expériences de projection encore plus évolués. Le module devait aussi permettre d'ouvrir la technologie de \texttt{RealityTech} à un plus grand nombre d'utilisateur du fait de la notoriété l'appréciation d'Unity dans son domaine. 